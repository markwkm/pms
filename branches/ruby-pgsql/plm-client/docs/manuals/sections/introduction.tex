\section{Patch Lifecycle Manager}
This manual covers the Patch Lifecycle Manager (\emph{PLM}).  The PLM is a system
for managing not only patches against a source tree but meta-data concerning
those patches as well.

The PLM can also run some limited tests against code submitted to the system to 
look for errors in code submissions.

\subsection{Patch Formats}
The format for patches is the accepted standard \emph{diff(1)} format.  The PLM
is not affected by the tools used to control the management of the source code
as long as patches in \emph{diff(1)} format can be generated.

Example:
\begin{verbatim}
diff -urN file.orig file.changed > file_version.patch
\end{verbatim}

\subsection{Scope}
\index{SCCS}
Most current SCCS systems allow the import and export of patches in diff format.
This common functionality allows the PLM to act as a sort of glue, enabling a
command presentation format separate from the software and formats used to manage 
the source code during development.

The PLM does not attempt to take on the role of a traditional SCCS system and 
considers managing the actual changes to individual source code files to be outside
it's area of interest.  The PLM is not well suited for SCCS work but is very well
suited as a value add to an environment where traditional SCCS tools are in place.

\subsection{Filters}
\index{Filters}
Filters are one of the powerful portions of the PLM.  A filter is essentially a 
scripted test that is run against a patch.  A filter could do anything from 
test the correct application of a patch to compile the resulting source.  A filter
could even be written to run a static profiler, lint analysis or other static analysis.
The one restriction that the filter environment has is the system cannot 
be rebooted as part of the filter test sequence.

\section{License}
\subsection{The Patch Lifecycle Manager software}
Copyright \copyright 2002-2003 Open Source Development Lab

This program is free software; you can redistribute it and/or modify
it under the terms of the GNU General Public License as published by
the Free Software Foundation; either version 2 of the License.

This program is distributed in the hope that it will be useful,
but WITHOUT ANY WARRANTY; without even the implied warranty of
MERCHANTABILITY or FITNESS FOR A PARTICULAR PURPOSE.  See the
GNU General Public License for more details.

You should have received a copy of the GNU General Public License
along with this program; if not, write to the Free Software
Foundation, Inc., 675 Mass Ave, Cambridge, MA 02139, USA.

\subsection{Changes}
Please send suggested changes to the maintainer at testdev@osdl.org.

