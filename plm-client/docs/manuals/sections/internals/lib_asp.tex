\subsection{Remote Procedure Calls}
The PLM::RPC:: module space contains all the methods which are called remoyely from any of the various clients via the call 'ASP', which takes care of the SOAP transport.  It has the following sub-modules:


\begin{itemize}
\item Command.pm
\item Filter.pm
\item Patch.pm
\item PrivateServer.pm
\item Server.pm
\item Software.pm
\item Supervisor.pm
\item User.pm
\end{itemize}

The modules CommandSet.pm, Filter.pm, Note.pm, Patch.pm, Software.pm, Supervisor.pm, User.pm set up functions which are meant to be called from Supervisors or plm clients and executed on the ASP (web) server.  They do the updating of and getting information from the database.  As of version 1.3.0 they are called via SOAP in general, although they may also be called directly from the PLM/Web modules.  When called in this manner, they should be called via the arrow (->) syntax in order to account for the package name argument.

The modules PrivateServer.pm and Server.pm are almost identical, and include the full set of the remote calls necessary for the secure and unsecure clients respectively.  The only real difference between PrivateServer.pm and Server.pm is that the functions which require security are only included through the PrivateServer.pm.  PrivateServer.pm has no security of its own, but is named such to let the admin know that security should be enabled THROUGH THE WEB SERVER CONFIGURATION.  This makes it easy to include all modules from the CGIs that receive the SOAP calls, plm\_private\_server.pl and plm\_server.pl.

\subsubsection{CommandSet.pm}
\index{PLM::RPC::CommandSet.pm}

\begin{verbatim}
command_set_get_content()
\end{verbatim}
\index{PLM::RPC::CommandSet.pm!command\_set\_get\_content()}
Select a list of install or build commands from the database and return a reference to the list.  Inputs are software type, patch ID, and command set type.

\subsubsection{Filter.pm}
\index{PLM::RPC::Filter.pm}

\begin{verbatim}
filter_request_by_patch()
\end{verbatim}
\index{PLM::RPC::Filter.pm!filter\_request\_by\_patch()}
Grab a list of objects representing the filter requests against a patch.  Input is the patch ID.


\begin{verbatim}
note_delete()
\end{verbatim}
\index{PLM::RPC::Note.pm!note\_delete()}

\begin{verbatim}
note_get()
\end{verbatim}
\index{PLM::RPC::Note.pm!note\_get()}


\subsubsection{Patch.pm}
\index{PLM::RPC::Patch.pm}

\begin{verbatim}
patch_add()
\end{verbatim}
\index{PLM::RPC::Patch.pm!patch\_add()}
Instantiate a patch object do some checks, add a new patch (to database) if all okay.

\begin{verbatim}
patch_delete()
\end{verbatim}
\index{PLM::RPC::Patch.pm!patch\_delete()}
\begin{verbatim}
patch_can_delete()
\end{verbatim}
\index{PLM::RPC::Patch.pm!patch\_can\_delete()}
\begin{verbatim}
patch_get()
\end{verbatim}
\index{PLM::RPC::Patch.pm!patch\_get()}
\begin{verbatim}
patch_add_depend()
\end{verbatim}
\index{PLM::RPC::Patch.pm!patch\_add\_depend()}
\begin{verbatim}
patch_delete_depend()
\end{verbatim}
\index{PLM::RPC::Patch.pm!patch\_delete\_depend()}
\begin{verbatim}
patch_find_by_name()
\end{verbatim}
\index{PLM::RPC::Patch.pm!patch\_find\_by\_name()}
This method returns the patch ID.  Input is a string which should contain all or part of the patch name.
\begin{verbatim}
patch_get_software_name()
\end{verbatim}
\index{PLM::RPC::Patch.pm!patch\_get\_software\_name()}
This method returns the name of the software repository associated with a given patch.  Input is the patch ID.
\begin{verbatim}
search_sanity_check()
\end{verbatim}
\index{PLM::RPC::Patch.pm!search\_sanity\_check()}
\begin{verbatim}
build_apply_list()
\end{verbatim}
\index{PLM::RPC::Patch.pm!build\_apply\_list()}
\begin{verbatim}
get_applies_list()
\end{verbatim}
\index{PLM::RPC::Patch.pm!get\_applies\_list()}
\begin{verbatim}
patch_search()
\end{verbatim}
\index{PLM::RPC::Patch.pm!patch\_search()}
\begin{verbatim}
patch_get_info()
\end{verbatim}
\index{PLM::RPC::Patch.pm!patch\_get\_info()}

