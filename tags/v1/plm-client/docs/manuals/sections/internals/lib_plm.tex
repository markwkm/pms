\subsection{PLM}
The PLM::PLM:: module space has the following sub-modules:

\begin{itemize}
\item Filter.pm
\item FilterRequest.pm
\item FilterRequestState.pm
\item FilterType.pm
\item Note.pm
\item Patch.pm
\item PatchACL.pm
\item PatchACL\_to\_User.pm
\item Software.pm
\item User.pm
\end{itemize}

All of the modules in the PLM::PLM:: space are children of the PLM::PLM object
and receive all of the Database routines from there.  The PLM::PLM object inherits from PLM::Object for the data storage.  This gives it a 'data' attribute, whichcontains the hash for the 'active' data, which can be accessed through the routines 'getElementValue', 'setElementValue', and has 'addElement' for initialising hash.

\subsubsection{Filter.pm}
\index{PLM::PLM::Filter.pm}
Provides an interface to the base filter object.

\begin{verbatim}
new()
\end{verbatim}
\index{PLM::PLM::Filter.pm!new()}
Instantiates a new object of type PLM::Filter.

\begin{verbatim}
debug( $value )
\end{verbatim}
\index{PLM::PLM::Filter.pm!debug()}
Sets the debug level for this module.

\begin{verbatim}
add( $xml_ref )
\end{verbatim}
\index{PLM::PLM::Filter.pm!add()}
Creates a new object reference, backed by the database.  If the \$xml\_ref is
included, then the new object starts out with the values from that scheme.  If
not, then the regular database defaults for each field apply.

\subsubsection{FilterRequest.pm}
\index{FilterRequest.pm}
Provides an interface to the filters request listing.

\begin{verbatim}
new()
\end{verbatim}
\index{PLM::PLM::FilterRequest.pm!new()}
Instantiates a new object of type PLM::FilterRequest.

\begin{verbatim}
debug( $value )
\end{verbatim}
\index{PLM::PLM::FilterRequest.pm!debug()}
Sets the debug level for this module.

\begin{verbatim}
add( $xml_ref )
\end{verbatim}
\index{PLM::PLM::FilterRequest.pm!add()}
Creates a new object reference, backed by the database.  If the \$xml\_ref is
included, then the new object starts out with the values from that scheme.  If
not, then the regular database defaults for each field apply.
