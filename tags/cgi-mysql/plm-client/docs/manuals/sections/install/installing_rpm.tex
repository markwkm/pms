\section {Installing the RPM}
\index{Install}

\subsection{RPM Dependencies}
\index{Install!RPM Dependencies}
The dependencies for the PLM depend on what type of host you are installing the RPM on (aka, what it's going to be used for)
In the future we plan to split out the RPM into the various uses and have rpm handle the dependency issues for us.

For CVS access, the following perl modules are required:  IO-Tty-1.02 (for Pty), Class-Accessor-0.19, IPC-Run-0.78, Cvs-0.06.  (The version numbers are just what I used, maybe older ones will work, except for Cvs where only 0.06 or better will work).


\subsection{Perl Dependencies}
\index{Install!Perl Dependencies}
For a PLM Supervisor client host, you only need the Mail::Internet perl module.

\subsection{Configuration}
The configuration file for the PLM is /etc/plm.cfg.  Each type of server/client will have different portions of that file that are important to it.  Those will be listed under their own configuration sections.

\subsection{Bug Workarounds}
\index{Install!Bug Workarounds}
The following bug workarounds are suggested:

\begin{itemize}
\item The file /var/log/plm.log must be created and writable by whatever user your PLM scripts will be run as.  In the future we plan to move all the logging to a syslog USER* facility.
\item You may need to force-nodeps to get the RPM to install
\end{itemize}
