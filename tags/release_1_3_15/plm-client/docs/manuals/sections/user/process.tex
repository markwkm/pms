\section{Process}
\subsection{Lifecycle of a Patch}
The lifecycle of a patch involves numerous stages:

\begin{itemize}
\item Creation of the patch (in diff format)
\item Submission to the PLM 
\item Filter application \& reporting
\item Testing (http://www.osdl.org/stp/)
\end{itemize}

\subsection{Base Version}
\index{Base Version}
A base version is special in that it does not have any content stored in 
PLM and does not apply to any parent version.  Instead the PLM holds an 
internal pointer to where the tarball can be retrieved by utilities to be
used as the base to apply other patches onto.

The PLM has a utility that will auto-submit software versions from a web 
accessible directory structure.  These patches are added with enough 
meta-data that other utilities will be able to find the correct .tar.gz 
or .tar.bz2 file when attempting to build the source tree.

Users cannot add base versions through any of the standard interfaces.  
Currently, they may be pulled from a web-based source repository.  The 
head of this
repository is configured in the plm\_source table.  Specific directions
are then set up in the 'plm\_source\_sync' table.  This is a bit 
complicated and 
needs to be done by an admin.  The same system can be used to pull 
patches posted at the repository.

Formerly, a base version had to be created by packaging up a full source 
tree and placing it in the local directory structure that was also 
available via web.  Now, you can point at a remote URL, but keep in mind 
that this system will have the frailties of using a remote sight and 
of not controlling when files may be rearranged.  The tar archive will be 
pulled from this location every time it is accessed.  This may be best for 
testing or for systems that are not heavily used.

At the OSDL we choose to have a local mirror of the Linux Kernel ftp archive.  
When a new kernel is released, the mirror places it in the local repository.  
Then the utility notices the new version and adds the new base version to 
the PLM repository.  This is a completely automatic procedure for us.

\subsection{Apply Trees}
\index{Apply Tree}
A patch tree can be created by applying patches to base versions or previously
submitted patches.  Note that a patch can only apply to a single lower version.  
The tree can be walked to find the base version.  Using this tree structure the
back-end can build a full source tree with applied patches to use for testing.

After the initial submission to the PLM, the patch can be used as a parent
in the applies tree of another patch.  When this happens, a patch may no 
longer be deleted from the system without first deleting all of the patches 
that depend on it.

\subsection{Submission}
\index{Patch Submission}
Submission to the PLM can happen through a number of interfaces.  The basic
information that must be provided about the patch at submission time is:

\begin{itemize}
\item Name of patch
\item Name or ID of version that patch applies to
\item Actual patch content
\end{itemize}

As you can see the initial submission does not contain a large amount of 
information regarding the patch.  Most of the information the PLM keeps 
regarding patches is gathered by the system through filter runs and various
tracking information such as submission date and a md5sum of the patch.

After patch submission, a series of filters (if available) are automatically 
queued to run against the patch.  The most basic filter we run is to test
that the patch actually applies to the version specified for it.  This 
is the minimal check that should be run on all patches submitted.

\subsection{Filters}
\index{Filters}
A filter is a script that does 'something' with a patch or against the 
full source tree created from a patch applies tree.  

Filters can be created to handle any task that does not require rebooting the 
machine the filter is running on.  Some of the filters we use at the OSDL to
check Linux Kernel patches include patch application checks as well as 
multiple compilation checks for multiple architectures.  The filters have the
ability to configure a kernel for compilation, execute the compilation and
verify a correct completion.  Another compilation test counts the build warnings 
and errors and reports on those instead of the basic pass/fail.

The output of filter runs is available through the web interface on the Patch
Info page.

Filters are run automatically by the back-end and users will receive an email 
notifying them when all of the filters against the patch are complete.

\subsection{Testing}
\index{Testing}
The PLM was created to service the need of the Scalable Test Platform.  The STP
had used CVS for it's patch submissions and required users to import entire CVS
versions of the source tree in a long and painful process.

After a patch is in the PLM, if there is an associated STP setup like there is
at the OSDL, additional testing can happen there.  The STP has the ability to 
run stress tests, workloads, regression tests in a much more flexible environment 
than the PLM filters.  STP tests start out with a complete OS install and are able
to deal with system crashes and reboots during testing.

Some software (such as kernels) cannot be installed without a reboot.  These 
software items cannot be fully tested within the PLM filter framework because 
filters cannot reboot the machine to install the software.  These types of software
packages can instead be tested by the Scalable Test Platform which integrates with
the PLM to provide additional testing flexibility.

\subsection{Patch Archive}
\index{Patch Archive}
The PLM will store patches until they are deleted.  There is no automatic purge operation.

In the future, we may look at ways of decreasing the amount of storage required 
by the back-end. The current .bz2 format does keep the individual patches small but the 
uncompress time slows things down.

Patches can be deleted by the user who submitted them.  When a user is logged in and a
patch can be deleted, a link to do so will be on the Patch Info page.  Patch deletion 
can only happen through the web interface.  If there are any other patches in the system 
that apply to the patch a user wishes to delete, those dependent patches must first be 
deleted.  If they are owned by other users, the user will not be able to resolve the
dependencies and will be unable to delete the patch from the system.

There is no automatic method for deleting STP test requests that refer to a patch ID 
that has been deleted.  It is for this reason we suggest not deleting patches if you have 
requested testing through the STP system.
