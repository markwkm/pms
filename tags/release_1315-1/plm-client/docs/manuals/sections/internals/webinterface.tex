\section{Web Interface}
\subsection{Web Application}
\index{Web Interface!Internals!Web Application}
The PLM web server application is served through a single cgi-bin program called \emph{plm}.

The \emph{plm} script makes use of the following standard libraries libraries:
\begin{itemize}
\item MIME::Base64
\item Fcntl
\item POSIX
\item CGI
\item CGI::Carp
\end{itemize}

The \emph{plm} script makes use of the following PLM libraries:
\begin{itemize}
\item PLM
\item PLMClient
\item PLM::Web::General
\item PLM::Web::Session
\item PLM::Web::Userpage
\item PLM::Web::Addpatch
\item PLM::Web::Patch
\end{itemize}

Web pages are organized as "modules" or "content sections".  Two of the available modules
are handled directly by the \emph{plm} script:

\begin{itemize}
\item content\_home
\item content\_filter\_output
\end{itemize}

The \emph{plm} script parses the new HTTP request from stdin and hands control
off to the correct content\_* function depending on the value of the \emph{module}
URL option.  If no module option was given, the content\_home page is served.  If
an invalid module is given a security warning is sent to syslog in the form of a 
CGI::Carp warn() message.

\subsection{ASP Server}
\index{Web Interface!Internals!ASP Server}

The regular interface for RPC calls is provided by the \emph{plm\_server.pl} script.

Authorization is handled on a as-need basis for function calls through this interface.
SSL is used as the encryption method for calls coming in through this interface.

The \emph{plm\_server.pl} script makes use of the following standard libraries:
\begin{itemize}
\item CGI::Carp
\end{itemize}

The \emph{plm\_server.pl} script makes use of the following PLM libraries:
\begin{itemize}
\item PLM::RPC::Server
\item PLM::PLM
\end{itemize}

The \emph{plm\_server.pl} script is a short one that
handles the SOAP calls.

\subsection{Private ASP Server}
\index{Web Interface!Internals!Private ASP Server}

A sub-set of RPC calls are available to trusted clients 
only through the \emph{plm\_private\_server.pl} script.  These are for use from the Supervisors.

Authorization should happen at the Web Server level because there is *zero* 
authentication built into this script.

The \emph{plm\_server.pl} script makes use of the following standard libraries:
\begin{itemize}
\item CGI::Carp
\end{itemize}

The \emph{plm\_server.pl} script makes use of the following PLM libraries:
\begin{itemize}
\item PLM::PLM
\item PLM::RPC::PrivateServer
\end{itemize}

The \emph{plm\_private\_server.pl} script is a short one that 
handles the SOAP calls.

\subsection{getpatch}
\index{Web Interface!Internals!getpatch}

The \emph{getpatch} script provides an easy URL link for downloading and 
viewing patches.  The hope is that the URL can be included in emails with
a logical formatting.

The \emph{getpatch} script makes use of the following standard CGI library 
as well as the main \emph{PLM} PLM library.

The URL parameter \emph{id} specifies the patch ID to retrieve.

The following ID extensions are supported:
\begin{itemize}
\item{.bzip2} Compressed download
\item{.html}  Pretty Printed formatting of text
\item{none}   Regular plain text download
\end{itemize}

The \emph{Content-Disposition: attachment; filename=} header is included to
tell the client that the data being downloaded should be saved to a file.

The \emph{Content-Type: application/octet-stream} header is included to tell
the client the same thing in case it didn't get it the first time.

There is no authorization on patch downloads.  If the patch exists, users 
can download it.

\subsection{PLM Report}
\index{Web Interface!Internals!plm\_report.pl}

\subsection{PLM Report Results}
\index{Web Interface!Internals!plm\_report\_results.pl}
