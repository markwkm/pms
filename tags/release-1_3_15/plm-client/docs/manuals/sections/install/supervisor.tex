\section {Configuring a Supervisor host}
\index{Configure!Supervisor}
The supervisor script controls the execution of the filters.  

\subsection{Multiple Supervisors}
\index{Configure!Supervisor!Multiple}
If your filters are single-threaded, you can run one supervisor per CPU on the machine.
If not, then you will want to take scaling issues into account when deciding how many 
supervisors to run on each machine.  Each supervisor will require it's own scratch space.

\subsection{Scratch Space}
\index{Configure!Supervisor!Scratch Space}
The scratch space is where the supervisor downloads the software, applies patches, builds
the software and applies the filters.  Having multiple supervisors run in the save 
scratch space will cause major problems.  To avoid this, the supervisor script takes a
'lockfile' command line option.  Pick a filename (such as PID) and use it for every
supervisor on the machine to avoid accidental issues.

\subsection{Supervisor Configuration}
\index{Configure!Supervisor!Configuration}
The supervisor hosts will also act as clients, so in addition to the following configuration parameters for a client should be set up. The important entries in the /etc/plm.cfg file for the asp\_supervisor.pl script are:

\begin{itemize}
\item ccache\_dir: Specifies where to store the ccache work files
\item supervisor\_sleep: Number of seconds to sleep between polling for new filters
\item filter\_type: Types of filters this supervisor can run (: delimited)
\end{itemize}

\subsection{Cron Job}
\index{Configure!Supervisor!Cron Job}
The cronjob for each supervisor you wish to run on the machine should do the following two things:

\begin{enumerate}
\item Change into the scratch
\item Launch the asp\_supervisor.pl script
\item Provide the asp\_supervisor.pl script with the lock filename
\item Redirect the output from asp\_supervisor.pl script to somewhere sane
\end{enumerate}

Examples:
\begin{verbatim}
*/15 * * * * cd /home/plm/scratch/1 && asp\_supervisor.pl PID 1>> LOG 2>> LOG &
*/15 * * * * cd /home/plm/scratch/2 && asp\_supervisor.pl PID 1>> LOG 2>> LOG &
\end{verbatim}

Note: You may need the full pathname to the asp\_supervisor.pl script.  
(/usr/local/bin/asp\_supervisor.pl)

\subsection{Configuration Steps}
We suggest the following steps for setting up a supervisor system:
\begin{enumerate}
\item Perform the regular steps from 'Installing the RPM'
\item Add a 'plm' user to the system
\item Create /home/plm/scratch/[1..CPU] directories for scratch space
\item If you have multiple disks, spread the actual scratch directories around and use symlinks
\item Configure the /etc/plm with the proper supervisor-specific options
\item Configure one cron job for each scratch directory
\item Make a link called /usr/local/bin/gcc to /usr/bin/ccache
\item Install "server-dead.sh" in /home/plm.  Make sure the path is set right in the script.  Set it up as a cron job hourly.
\item (this is a bug workaround) chmod +x /usr/local/bin/asp\_supervisor.pl
\end{enumerate}
