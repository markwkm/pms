\section {Configuring an Administrative Client}

\subsection{Administrative Client Configuration}
\index{Configure!Administrative!Configuration}
The important entries in the /etc/plm.cfg file for the various administrative scripts are:

\begin{itemize}
\item PLMClient\_uri: Set to 'Base' (the name of a Perl module)
\item PLMClient\_proxy: URL or relative URL to 'plm\_private\_server.pl'
\item getpatch\_url: URL to 'getpatch' CGI script.
\item patch\_replication\_target: For table plm\_patch replication we need a target database.  This is because it is different than the database for the remote 'user' table.
\end{itemize}

\subsection{Configuration Steps}
The following steps are for setting up an administrative client system:
\begin{enumerate}
\item Perform the regular steps from 'Installing the RPM'
\item Add a 'plm' user to the system
\item Configure the /etc/plm with the proper administrative client-specific options.
\end{enumerate}
